\documentclass[11pt]{preprint}
\usepackage[utf8]{inputenc}
\usepackage[english]{babel}
\usepackage{amssymb}
\usepackage{amsmath}
\usepackage{hyperref}
\usepackage{breakurl}
\usepackage{mhenvs}
\usepackage{mhequ}
\usepackage{mhsymb}	
\usepackage{booktabs}
\usepackage{tikz}
\usepackage{mathrsfs}
\usepackage{times}
\usepackage{microtype}
\usepackage{comment}
\usepackage{slashed}
\usepackage{mathtools}
%\usepackage{wasysym}
\usepackage{centernot}
\usepackage{leftidx}
\usepackage{accents}
\usepackage{arydshln}
%\usepackage{showkeys}
\usepackage{footnote}
\makesavenoteenv{tabular}
\usepackage{enumitem}
\usepackage{stackrel}
\usepackage{halloweenmath}
\usepackage{longtable}%\usepackage{scalerel}
\usepackage{array}
\usepackage{cprotect}
\usepackage{xstring}
%\usepackage{wasysym}
%\usepackage{centernot}
\usepackage{ulem}
\usepackage{stmaryrd}
\usepackage{titlesec}
\usepackage{authblk}
\usepackage{mleftright}
%\usepackage[margin=1in]{geometry}


%%%%Commutative Diagram Package%%%%
\usepackage[cmtip,arrow]{xy}
\usepackage{lamsarrow}
\usepackage{pb-diagram}
\usepackage{pb-lams}
\usepackage{pb-xy}
%%%%%%%%%%%%%%%%%%%%%%%%%%%%%%%%%%%



\setlength{\marginparsep}{2mm}
\setlength{\marginparwidth}{4cm}

\renewcommand{\ttdefault}{lmtt}

\DeclareSymbolFont{timesoperators}{T1}{ptm}{m}{n}
\SetSymbolFont{timesoperators}{bold}{T1}{ptm}{b}{n}

\makeatletter
\renewcommand{\operator@font}{\mathgroup\symtimesoperators}
\makeatother



\colorlet{symbolsgrey}{blue!30!black!50}
\colorlet{testcolor}{green!60!black}
\definecolor{purple}{rgb}{0.55,0.05,0.8}
\definecolor{symbols}{rgb}{0.55,0.05,0.8}

\usetikzlibrary{shapes.misc}
\usetikzlibrary{shapes.symbols}
\usetikzlibrary{shapes.geometric}
\usetikzlibrary{snakes}
\usetikzlibrary{decorations}
\usetikzlibrary{decorations.markings}

\usetikzlibrary{calc}
\usetikzlibrary{external}


\newtheorem{assumption}[lemma]{Assumption}
\newtheorem{example}[lemma]{Example}
\newtheorem{defn}[lemma]{Definition}


\let\oldskull\skull
\def\skull{\mathord{\oldskull}}


\def\sol{{\mathop{\mathrm{sol}}}}
\def\init{{\mathop{\mathrm{init}}}}
\def\pr{\textcolor{purple}}
\def\expan{\mcb{H}} 

\newcommand{\hooklongrightarrow}{\lhook\joinrel\longrightarrow}
\newcommand{\longrightharpoonup}{\relbar\joinrel\rightharpoonup}

\def\monom{\Theta}

\def\restr{\mathbin{\upharpoonright}}

\def\Tran{\mathrm{Tran}}
\def\wotimes{\mathbin{\widehat{{\otimes}}}}
\def\totimes{\mathbin{\widetilde{{\otimes}}}}
\def\snabla{\slashed{\nabla}}

\DeclareMathAlphabet{\mathbbm}{U}{bbm}{m}{n}

\overfullrule=3mm
\marginparwidth=3.3cm

\DeclareFontFamily{U}{BOONDOX-calo}{\skewchar\font=45 }
\DeclareFontShape{U}{BOONDOX-calo}{m}{n}{
  <-> s*[1.05] BOONDOX-r-calo}{}
\DeclareFontShape{U}{BOONDOX-calo}{b}{n}{
  <-> s*[1.05] BOONDOX-b-calo}{}
\DeclareMathAlphabet{\mcb}{U}{BOONDOX-calo}{m}{n}
\SetMathAlphabet{\mcb}{bold}{U}{BOONDOX-calo}{b}{n}
%\DeclareMathAlphabet{\mathbcalboondox}{U}{BOONDOX-calo}{b}{n}

\setlist{noitemsep,topsep=4pt}



\makeatletter % Stolen from the internet to make a fat \cdot which isn't as fat as a \bullet
\newcommand*{\bigcdot}{}% Check if undefined
\DeclareRobustCommand*{\bigcdot}{%
  \mathbin{\mathpalette\bigcdot@{}}%
}
\newcommand*{\bigcdot@scalefactor}{.5}
\newcommand*{\bigcdot@widthfactor}{1.15}
\newcommand*{\bigcdot@}[2]{%
  % #1: math style
  % #2: unused
  \sbox0{$#1\vcenter{}$}% math axis
  \sbox2{$#1\cdot\m@th$}%
  \hbox to \bigcdot@widthfactor\wd2{%
    \hfil
    \raise\ht0\hbox{%
      \scalebox{\bigcdot@scalefactor}{%
        \lower\ht0\hbox{$#1\bullet\m@th$}%
      }%
    }%
    \hfil
  }%
}
\makeatother


\def\symbol#1{\textcolor{symbolsgrey}{#1}}
\def\1{\mathbf{\symbol{1}}}
\def\X{\symbol{X}}
\def\enorm#1{| #1|_{\eps}}
\def\snorm#1{\Vert #1\Vert_\s}
\def\sabs#1{| #1|_\s}
\newcommand{\verteq}{\rotatebox{90}{=}}
\newcommand{\cev}[1]{\reflectbox{\ensuremath{\vec{\reflectbox{\ensuremath{#1}}}}}}
\newcommand{\veccev}[1]{\overset{\text{\tiny$\leftrightarrow$}}{#1}}
\def\PPsi{\symbol{\Psi}}
\def\hPsi{\hat{\symbol{\Psi}}}
\def\sXi{\symbol{\Xi}}
\def\I{\symbol{\CI}}
\def\J{\symbol{\CJ}}
\def\Projection{\mathbf P}

\def\bone{\mathbf{1}}

\def\higgsvec{\mathbf{V}}

\def\Cst{{}_{\eps}C}

\def\smooth{\mcb{C}^\infty}

\let\p\partial
\def\bB{\boldsymbol{B}}
\def\bPsi{\boldsymbol{\Psi}}
\def\tbPsi{\tilde{\boldsymbol{\Psi}}}
\def\id{\mathrm{id}}
\def\PPi{\boldsymbol{\Pi}}
\def\hPPi{\rlap{$\hat{\phantom{\PPi}}$}\PPi}
\def\bbrho{\bar{\boldsymbol{\rho}}}
\def\brho{\boldsymbol{\rho}}

\def\Lab{\mathfrak{L}}
\def\sco{f}

\def\inv{\textnormal{\tiny inv}}
\def\op{\textnormal{\tiny op}}
\def\edge{\textnormal{\tiny edge}}
\def\edge{\textnormal{\tiny node}}

\def\can{\textnormal{\scriptsize can}}
\def\sym{\textnormal{\tiny \textsc{sym}}}
\def\symg{\textnormal{\tiny \textsc{symg}}}
\def\gsym{\textnormal{\tiny \textsc{gsym}}}
\def\BPHZ{\textnormal{\tiny \textsc{bphz}}}
\def\YMH{\textnormal{\tiny \textsc{ymh}}}
\def\YM{\textnormal{\tiny \textsc{ym}}}
\def\YMHG{\textnormal{\tiny \textsc{ymhg}}}
\def\Higgs{\textnormal{\tiny {Higgs}}}
\def\Gauge{\textnormal{\tiny {Gauge}}}
\def\even{\textnormal{\tiny even}}
\def\odd{\textnormal{\tiny odd}}
\def\rand{\mbox{\tiny rand}}
\def\Div{\mathrm{Div}}
\def\Dom{\mathrm{Dom}}
\def\DDom{\mathcal{D}}
\def\A{\textnormal{A}}
\def\h{\textnormal{h}}
\def\lead{\textnormal{\tiny lead}}


\def\moll{\chi}
\def\Hol#1{#1\textnormal{-H{\"o}l}}
\def\var#1{#1\textnormal{-var}}
\def\hol{\textnormal{hol}}
\def\gr#1{#1\textnormal{-gr}}
\def\rp#1{#1\textnormal{-rp}}
\def\state{\mathcal{S}}
\def\init{\mathcal{I}}
\let\ymhflow\CF
\def\flow{\CE}

\def\fancynorm#1{{\talloblong #1 \talloblong}}
\def\heatgr#1{{|\!|\!| #1 |\!|\!|}}

\newcommand{\mrd}{\mathrm{d}}
\newcommand{\floor}[1]{\lfloor #1 \rfloor}
\newcommand{\roof}[1]{\lceil #1 \rceil}


\colorlet{darkblue}{blue!90!black}
\colorlet{darkgreen}{green!50!black}
\let\comm\comment
\def\martin#1{\comm[darkred]{MP: #1}}
\def\aksh#1{\comm[darkblue]{AD: #1}}


\long\def\ajayText#1{{\color{darkred}Ajay:\ #1}}
\long\def\martinText#1{{\color{darkblue}Martin:\ #1}}
\long\def\haoText#1{{\color{darkgreen}Hao:\ #1}}
\long\def\ilyaText#1{{\color{brown}Ilya:\ #1}}

\let\M\CM
\let\D\CD
\def\s{\mathfrak{s}}
\def\KK{\mathfrak{K}}
\def\be{\mathbf e}
\def\poly #1{{#1}_{\mathrm{poly}}}
\def\onorm#1{| #1|_0}

\def\reg{\mathrm{reg}}
\def\rhs{\mathrm{rhs}}

\def\Func{\mathbf{F}}
\let\eref\eqref
\def\proj{\mathbf{p}}
\newcommand{\e}{\varepsilon}
\newcommand{\V}  {\overline{V}}
\def\K{\mathfrak{K}}
\def\DeltaM{\Delta^{\!M}}
\def\hDeltaM{\hat \Delta^{\!M}}
\def\Ren{\mathscr{R}}
\def\MM{\mathscr{M}}
\def\TT{\mathscr{T}}
\def\DD{\symbol{\mathscr{D}}}
\def\RR{\mathfrak{R}}
\def\GGamma#1{\Gamma_{\!#1}}
\def\bGGamma#1{\bar \Gamma_{\!#1}}
\def\bbar#1{\bar{\bar #1}}
\def\${|\!|\!|}
\def\Wick#1{\mathord{\kern0.1em{:}#1{:}\kern0.1em}}
\def\E{\mathbf{E}}
\def\T{\mathbf{T}}
\def\Cum{\mathbf{E}_c}
\def\P{\mathbf{P}}
\def\powerset{\mathscr{P}}
\def\div{\mbox{div}}
\def\curl{\mbox{curl}}
\def\d{\mbox{d}}
\def\-{\mbox{-}}
%\def\loop{C_U}

\def\r{\mathfrak{r}}
\def\v{\mathfrak v}
\def\l {\mathfrak l}
\def\f{\mathfrak{f}}
\def\labphi{\boldsymbol{\phi}}

\def\bXi{\boldsymbol{\Xi}}

\def\LieAlg{\mathfrak g}

%mathfraks
\newcommand{\mfu}{\mathfrak{u}}
\newcommand{\mfU}{\mathfrak{U}}
\newcommand{\mfB}{\mathfrak{B}}
\newcommand{\mfF}{\mathfrak{F}}
\newcommand{\mfI}{\mathfrak{I}}
\newcommand{\mfT}{\mathfrak{T}}
\newcommand{\mfn}{\mathfrak{n}}
\newcommand{\mfo}{\mathfrak{o}}
\newcommand{\mfe}{\mathfrak{e}}
\newcommand{\mfN}{\mathfrak{N}}
\newcommand{\mfL}{\mathfrak{L}}
\newcommand{\mfd}{\mathfrak{d}}
\newcommand{\mfa}{\mathfrak{a}}
\newcommand{\mfc}{\mathfrak{c}}
\newcommand{\mfA}{\mathfrak{A}}
\newcommand{\mfC}{\mathfrak{C}}
\newcommand{\mfR}{\mathfrak{R}}
\newcommand{\mfr}{\mathfrak{r}}
\newcommand{\mft}{\mathfrak{t}}
\newcommand{\mfm}{\mathfrak{m}}
\newcommand{\mfM}{\mathfrak{M}}
\newcommand{\mfS}{\mathfrak{S}}
\newcommand{\mfp}{\mathfrak{p}}
\newcommand{\mfMO}{\mathfrak{MO}}
\newcommand{\mfJ}{\mathfrak{J}}
\newcommand{\mfl}{\mathfrak{l}}
\newcommand{\mfz}{\mathfrak{z}}
\newcommand{\mfh}{\mathfrak{h}}
\newcommand{\mfq}{\mathfrak{q}}
\newcommand{\mfb}{\mathfrak{b}}
\newcommand{\mff}{\mathfrak{f}}
\newcommand{\mfD}{\mathfrak{D}}
\newcommand{\mfG}{\mathfrak{G}}
\newcommand{\mfO}{\mathfrak{O}}
\newcommand{\mfP}{\mathfrak{P}}
\newcommand{\mfg}{\mathfrak{g}}
\newcommand{\mfk}{\mathfrak{k}}
\newcommand{\mfZ}{\mathfrak{Z}}
\newcommand{\mfH}{\mathfrak{H}}
\newcommand{\mfK}{\mathfrak{K}}
%mathcals
\newcommand{\mcO}{\mathcal{O}}
\newcommand{\mcM}{\mathcal{M}}
\newcommand{\mcE}{\mathcal{E}}
\newcommand{\mcH}{\mathcal{H}}
%\newcommand{\G}{\mathcal{G}}
\newcommand{\U}{U}
\newcommand{\mcA}{\mathcal{A}}
\newcommand{\mcV}{\mathcal{V}}
\newcommand{\mcR}{\mathcal{R}}
\newcommand{\mcC}{\mathcal{C}}
\newcommand{\mcB}{\mathcal{B}}
\newcommand{\mcJ}{\mathcal{J}}
\newcommand{\mcS}{\mathcal{S}}
\newcommand{\mcU}{\mathcal{U}}
\newcommand{\mcL}{\mathcal{L}}
\newcommand{\mcT}{\mathcal{T}}
\newcommand{\mcI}{\mathcal{I}}
\newcommand{\mcF}{\mathcal{F}}
\newcommand{\mcY}{\mathcal{Y}}
\newcommand{\mcN}{\mathcal{N}}
\newcommand{\mcK}{\mathcal{K}}
\newcommand{\mcD}{\mathcal{D}}
\newcommand{\mcW}{\mathcal{W}}
\newcommand{\mcP}{\mathcal{P}}
\newcommand{\mcG}{\mathcal{G}}
\newcommand{\mcX}{\mathcal{X}}
\newcommand{\mcQ}{\mathcal{Q}}
\newcommand{\mcZ}{\mathcal{Z}}

\def\symset{\mcb{s}}

%mathbfs
\newcommand{\bu}{\mathbf{u}}
\newcommand{\mbo}{\mathbf{o}}
\newcommand{\mba}{\mathbf{a}}
\newcommand{\mbA}{\mathbf{A}}
\newcommand{\mbC}{\mathbf{C}}
\newcommand{\mbF}{\mathbf{F}}
\newcommand{\mbD}{\mathbf{D}}
\newcommand{\mbL}{\mathbf{L}}
\newcommand{\mbK}{\mathbf{K}}
\newcommand{\mbS}{\mathbf{S}}
\newcommand{\mbW}{\mathbf{W}}
\newcommand{\mbZ}{\mathbf{Z}}
\newcommand{\mbY}{\mathbf{Y}}
\newcommand{\mbX}{\mathbf{X}}
\newcommand{\mbM}{\mathbf{M}}
\newcommand{\mbn}{\mathbf{n}}
\newcommand{\mbm}{\mathbf{m}}
\newcommand{\mbj}{\mathbf{j}}
\newcommand{\mbi}{\mathbf{i}}
\newcommand{\mbk}{\mathbf{k}}
\newcommand{\mbb}{\mathbf{b}}
\newcommand{\mbp}{\mathbf{p}}
\newcommand{\mbg}{\mathbf{g}}


%mathrms
\newcommand{\ch}{\mathrm{c}}
\newcommand{\inte}{\mathrm{int}}
\newcommand{\exte}{\mathrm{ext}}
\newcommand{\ex}{\mathrm{ex}}
\newcommand{\loc}{\mathrm{loc}}
\newcommand{\CBH}{\mathrm{CBH}}
\newcommand{\ad}{\mathrm{ad}}
\newcommand{\Ad}{\mathrm{Ad}}
\newcommand{\End}{\mathrm{End}}
\newcommand{\Aut}{\mathrm{Aut}}
\newcommand{\Coll}{\mathrm{Coll}}
\newcommand{\Trace}{\mathrm{Tr}}
\newcommand{\Haus}{\mathrm{H}}



%mathscrs
\def\cF{\mathscr{F}}
\def\cX{\mathscr{X}}
\def\cC{\mathscr{C}}
\def\cA{\mathscr{A}}
\def\cG{\mathscr{G}}
\def\cH{\mathscr{H}}
\def\cS{\mathscr{S}}
\def\cD{\mathscr{D}}
\def\cE{\mathscr{E}}
\def\cB{\mathscr{B}}
\def\cK{\mathscr{K}}
\def\cT{\mathscr{T}}
\def\cZ{\mathscr{Z}}


\def\ST{\mathscr{T}}
\newcommand{\VERT}{\vert\!\vert\!\vert}

\DeclareMathOperator{\Ran}{Ran}
\DeclareMathOperator{\im}{im}
\DeclareMathOperator{\sgn}{sgn}


\def\divB{\nabla\!\cdot\! B}
\def\cdota{\!\cdot\!}
\def\cdota{\!\!\cdot\!\!}
\def\sig{\boldsymbol{\sigma}}

%\def\emptyset{\mathop{\centernot\ocircle}}
%\def\emptyset{{\centernot\Circle}}

\def\diam{\mathrm{diam}}
\def\act{\bigcdot}

\def\bUpsilon{\boldsymbol{\Upsilon}}
\def\bbUpsilon{\boldsymbol{\bar{\Upsilon}}}

\def\Set{\mathop{\mathrm{Set}}\nolimits}
\def\Op{\mathord{\mathbf{Op}}}
\def\Ob{\mathord{\mathbf{Ob}}}
\def\Ab{\mathord{\mathbf{Ab}}}
\def\Hom{\mathord{\mathrm{Hom}}}
\def\Homb{\mathord{\overline{\mathrm{Hom}}}}
\def\VHomb{\mathord{\overline{\mathrm{SHom}}}}
\def\Iso{\mathord{\mathrm{Iso}}}
\def\TStruc{\mathord{\mathrm{TStruc}}}
\def\SSet{\mathord{\mathrm{SSet}}}
\def\Cas{\mathord{\mathrm{Cas}}}
\def\Cov{\mathrm{Cov}}
\def\bCov{\mathbf{Cov}}

\def\Cech{\check{H}}
\def\Rham{H_{\mathrm{dR}}}

\DeclareMathOperator{\Cl}{C\ell}
\DeclareMathOperator{\Spin}{Spin}
\DeclareMathOperator{\SO}{SO}



\newcommand{\fpar}[2]{\frac{\partial #1}{\partial #2}}
\newcommand{\spar}[2]{\frac{\partial^2 #1}{\partial #2^2}}
\newcommand{\mpar}[3]{\frac{\partial^2 #1}{\partial #2 \partial #3}}
\newcommand{\tpar}[2]{\frac{\partial^3 #1}{\partial #2^3}}


\numberwithin{equation}{section}


\def\dash{\leavevmode\unskip\kern0.18em--\penalty\exhyphenpenalty\kern0.18em}
\def\slash{\leavevmode\unskip\kern0.15em/\penalty\exhyphenpenalty\kern0.15em}


\def\hstar{\mathbin{\hat{*}}}


\let\f\frac




%%%%%%%%%%%%%%%%%%%%%%%%%%%
%%%%%%%%%%%%%%%%%%%%%%%%%%%
%%%%%%%%%%%%%%%%%%%%%%%%%%%



\makeatletter

\DeclareRobustCommand{\TitleEquation}[2]{\texorpdfstring{\StrLeft{\f@series}{1}[\@firstchar]$\if%
b\@firstchar\boldsymbol{#1}\else#1\fi$}{#2}}

\makeatother



\begin{document}
%

\title{Complex Manifolds and K\"ahler Geometry\\ Project 4}

\author[1, 2]{Akshunna S. Dogra\footnote{adogra@nyu.edu}}
\affil[1] {\small Department of Mathematics, Imperial College London}
\affil[2] {\small EPSRC CDT in Mathematics of Random Systems: Analysis, Modelling and Simulation}
\institute{Imperial College London}

\maketitle


\begin{abstract}
In this essay we go over the proof of the isomorphism between the $\check{\mathrm{C}}$ech cohomology
 \TitleEquation{\Cech(M;\R)}{2}
 and the de Rham cohomology \TitleEquation{\Rham(M;\R)}{2} 
 of a smooth manifold $M$ based on the exposition in \cite{GH78}.
\end{abstract}


\section{Short Introduction to Sheaves}

Let $(X,\mathcal T)$ be a topological space. We associate with this space the category $\Op(X)$ with its set of objects just being the open sets $\mathcal T$ and the morphisms $\Hom(U,V)$ either being solely the inclusion $U \hookrightarrow V$ if $U \subset V$ or otherwise $\emptyset$.

Further, let $\Ab$ denote the category of Abelian groups.

\begin{defn}
    Let $\mathcal F : \Op(X) \to \Ab$ be a contravariant functor. We call elements $\sigma \in \mathcal F(V)$ sections of $V$ and the image of the inclusion map $U \hookrightarrow V$ under the functor by $\sigma|_U$, the restriction of $\sigma$ to $U$. 
    
    We call such a functor $\mathcal F$, a \textit{pre-sheaf}, if $\mathcal F(\emptyset) = \mathbf 0$, the $0$-group.
\end{defn}

This definition encodes some of the basic notions one associates with local constructions on a manifold, such as the set of holomorphic functions on an open subset of a complex manifold, or the local vector fields on a manifold. However, these enjoy some further gluing properties. For this reason we add two requirements that allow us to speak of sheaves.

\begin{defn}
   A pre-sheaf $\mathcal F : \Op(X) \to \Ab$ is called a \textit{sheaf} if in addition it satisfies the following two requirements: 
   \begin{itemize}
     \item For all $U,V \in \mathcal T$ and all sections $\sigma \in \mathcal F(U)$ and $\tau \in \mathcal F(V)$ such that
     \[
        \sigma\big|_{U \cap V} = \tau \big|_{U \cap V}  
     \]
     there exists $\omega \in \mathcal F(U \cup V)$ s.t.\ 
     \[
        \sigma = \omega\big|_{U}, \qquad \tau = \omega\big|_{V}.  
     \]
     \item If $\sigma \in \mathcal F(U\cup V) $, s.t.\ $\sigma\big|_U = \sigma\big|_V = 0$, then $\sigma = 0$.
   \end{itemize}
\end{defn}

Let $G$ be a fixed Abelian group. One might think that an easy example of sheaf would be the constant sheaf that assigns to every open set $U$ (other than the empty set) the whole group $G$, i.e.\ $\mathcal F(U) = G$, with all the restriction maps being the identity. However, once the space $X$ has more than one connected one runs into trouble. For suppose that $X$ has two connected components $X_1 \cup X_2 = X$. Then for $g_1, g_2 \in G$ distinct one can set 
\[
  g_1 \eqqcolon \sigma \in \mathcal F(X_1), \qquad  g_2 \eqqcolon \sigma \in \mathcal F(X_2).  
\]
Since $X_1 \cap X_2 = \emptyset$ one trivially has
\[
  \sigma\big|_{X_1 \cap X_2}  = 0 = \tau\big|_{X_1 \cap X_2} 
\]
but there cannot be a $g \in \mathcal F(X) = G$ such that its restrictions are equal to $g_1$ and $g_2$ for
\[
  g\big|_{X_1} = g = g\big|_{X_2}.  
\]
This therefore only a pre-sheaf. The correct ``sheafification'' of the constant pre-sheaf is the constant sheaf which we shall present together with a couple of examples.

\begin{example}
  \begin{itemize}
    \item The constant sheaf $\underline{G}$ which assigns to $U \in \mathcal T$ the group $G^{n}$ where $n$ is the number of connected components of $X$ $U$ intersects. The restriction maps are the obvious ones. 
    \item Let $M$ be a smooth manifold. Let $\cA^p$ be the sheaf of smooth $p$-forms on $M$, i.e.\ for $U \subset M$, $\cA^p(U)$ is the (vector) space of smooth, real-valued $p$-forms defined on $U$.
    \item Analogously to above let $\cZ^p$ be the sheaf of closed $p$-forms on $M$.
    \item Let $(X,J)$ be a complex manifold. Let $\mathscr O$ denote the sheaf of holomorphic functions on $X$, i.e.\ for $U \subset X$, $\mathscr O(U)$ is the (vector) space of holomorphic functions defined on $U$.
  \end{itemize}
\end{example}

\begin{remark}
  Note that $\cA^0$ is simply the sheaf $\mathscr C^\infty$ of smooth functions on $M$ and $\cZ^0$ is the constant sheaf $\underline{\R}$.
\end{remark}

\section{Cohomology of Sheaves}

Let $\mathcal F$ be a sheaf on $X$ and $\mathcal U = \left(U_\alpha\right)_{\alpha \in I}$ a locally finite open cover, that is a cover s.t.\ for any $U_\alpha \in \mathcal U$ there exist only finitely many $U_\beta \in \mathcal U$ that are non-zero.

We define the chain complex of $\mathcal U$ and $\mathcal F$ to be given by the groups 
\[
  C^p(\mathcal U, \mathcal F) \eqdef \prod_{\substack{\alpha_0, \dots, \alpha_p \in I \\ \alpha_0 \neq \cdots \neq \alpha_p }} \mathcal F \left( U_{\alpha_0} \cap \cdots \cap U_{\alpha_p} \right)  
\]
and coboundary operator $d : C^p(\mathcal U, \mathcal F) \to C^{p+1}(\mathcal U, \mathcal F)$
\[
  \left(d \omega\right)_{\alpha_0 \cdots \alpha_{p+1}} =  \sum_{i = 0}^{p+1} (-1)^{i} \omega_{\alpha_1 \cdots \widehat{\alpha_i} \cdots \alpha_p} \big|_{U_{\alpha_0} \cap \cdots \cap U_{\alpha_{p+1}} }   
\]
where $\widehat{\alpha_i}$ denotes the lack of the index $\alpha_i$. Elements of $C^p(\mathcal U, \mathcal F)$ are called \textit{$p$-cochains}. A $p$-cochain $\sigma$ will be known as a \textit{cocylce} if $d\sigma = 0$, the set of which shall be denoted by $Z^p(\mathcal U, \mathcal F)$, and a \textit{coboundary} if there exists $p-1$-cochain $\tau$, s.t. $d\tau = 0$. 

It is a routine calculation to check that $d^2 = 0$, thus $d C^{p-1} \subset Z^p$ and one can define the cohomology group
\[
  H^p(\mathcal U, \mathcal F) \coloneqq \frac{Z^p(\mathcal U,\mathcal F)}{dC^{p-1}(\mathcal U,\mathcal F)}.
\]
However, this group strongly depends on the choice of open cover $\mathcal U$. Thus, we define the $\check{\mathrm{C}}$ech cohomology group of the sheaf $\mathcal F$ over $X$ to be the direct limit over all locally finite open covers of the above group, i.e.\ 
\[
  H^p(X,\mathcal F)  \coloneqq  \check H^p(X,\mathcal F)  \coloneqq \varinjlim_{\mathcal U}   H^p(\mathcal U, \mathcal F).
\] 
Of course, it is extremely hard to calculate the above direct limit in general. However, often there are conditions under which for a sufficiently fine yet still finite cover $\mathcal U$ of a sufficiently nice space $X$ one has 
\[
  H^p(X,\mathcal F)  \cong  H^p(\mathcal U, \mathcal F).
\]
We shall prove that this is the case in the following section for the $\check{\mathrm{C}}$ech cohomology of a manifold.

\begin{defn}[$\check{\mathrm{C}}$ech Cohomology of a Smooth Manifold]
  Let $M$ be a smooth connected manifold. We call the group $\check H^p(M,\R)$ the $p^{\text{th}}$ $\check{\mathrm{C}}$ech cohomology group of $M$.
\end{defn}



\section{De Rham's Theorem} 

We shall need the following two lemmate to show how one can calculate the $\check{\mathrm{C}}$ech cohomology group of a manifold.

\begin{lemma}
    For $p > 0$ and all $q\geqslant 0$, the $\check{\mathrm{C}}$ech cohomology groups of the sheaves $\cA^q$ are trivial, i.e. $\Cech^p(M,\cA^q) = 0$. 
\end{lemma}
\begin{proof}
  It is enough to prove that this holds for any locally finite cover $\CU$ of $M$. Let $\CU = \left( U_\alpha\right)_{\alpha \in I}$ be such a cover and let $\left(\rho_\alpha\right)_{\alpha \in I}$ be a corresponding subordinate, i.e.\ $\supp(\rho_\alpha) \subset U_\alpha$, partition of unity.

  Let $\omega \in Z^p(\CU, \cA^q)$ we have to find a $\psi \in C^{p-1}(\CU, \cA^q)$ such that $d \psi = \omega$. %To simplify the argument, we shall first assume that $ \supp \omega \subset U$ for some fixed $U \in \CU$.
  $\omega$ consists of a collection of maps 
  \[
      \omega_{\alpha_0 \cdots \alpha_p} : \bigcap_{i = 1}^n U_{\alpha_i} \longrightarrow \cA^q \left( \bigcap_{i = 1}^n U_{\alpha_i} \right)
  \]
  for any combination of $p$ distinct indices $\alpha_1, \dots, \alpha_p \in I$. For indices $\alpha_1, \dots, \alpha_{p-1} \in I$ we define $\psi$ to be
  \[
    \psi_{\alpha_0 \cdots \alpha_{p-1}} = \sum_{\beta \in I} \rho_\beta \omega_{\beta\alpha_0 \cdots \alpha_{p-1}}.
  \]
  We have to prove that $\omega = d \psi$. On $\bigcap_{i = 1}^p U_{\alpha_i}$
  \begin{align*}
    \left(d \psi\right)_{\alpha_0 \cdots \alpha_{p}} &= \sum_{i = 0}^p (-1)^i \psi_{\alpha_0 \cdots \widehat{\alpha_i} \cdots \alpha_p} \big|_{U_{\alpha_i}}  \\
    & = \sum_{i = 0}^p \sum_{\beta \in I} (-1)^i \rho_\beta \big|_{U_{\alpha_i}} \omega_{\beta\alpha_1 \cdots \widehat{\alpha_i} \cdots \alpha_p} \big|_{U_{\alpha_i}}  \\
    & = \sum_{i = 0}^p \sum_{\beta \in I} (-1)^i \rho_\beta \big|_{\bigcap_{j = 0}^p U_{\alpha_j}} \omega_{\beta\alpha_0 \cdots \widehat{\alpha_i} \cdots \alpha_p} \big|_{U_{\alpha_i}}  \\
    & =  \sum_{\beta \in I} \rho_\beta \big|_{\bigcap_{j = 0}^p U_{\alpha_j}}  \sum_{i = 0}^p (-1)^i \omega_{\beta\alpha_0 \cdots \widehat{\alpha_i} \cdots \alpha_p} \big|_{U_{\alpha_i}} \\
    & = \sum_{\beta \in I} \rho_\beta \big|_{\bigcap_{j = 0}^p U_{\alpha_j}}   \omega_{\alpha_0  \cdots \alpha_p} \big|_{U_{\beta}} \\
    & = \sum_{\beta \in I} \rho_\beta  \omega_{\alpha_0  \cdots \alpha_p} =\omega_{\alpha_0  \cdots \alpha_p} .
  \end{align*}
  where in the third line we used that on $ \bigcap_{i = 0}^n U_{\alpha_i}$ and all $i \in \{0,\dots, p\}$
  \[
      \rho_\beta \big|_{\bigcap_{j = 0}^p U_{\alpha_j}} \equiv   \rho_\beta\big|_{U_{\alpha_i}} ,
  \]
  in the forth the fact that $\omega \in Z^p$ and thus 
  \[
    \omega_{\alpha_0\cdots\alpha_p}\big|_{U_\beta} - \sum_{i = 0}^p (-1)^i \omega_{\beta\alpha_0 \cdots \widehat{\alpha_i} \cdots \alpha_p} \big|_{U_{\alpha_i}} = 0,
  \]
  in the last line that on $\bigcap_{i = 1}^n U_{\alpha_i}$ 
  \[
    \rho_\beta \big|_{\bigcap_{j = 0}^p U_{\alpha_j}}   \omega_{\alpha_1  \cdots \alpha_p} \big|_{U_{\beta}} \equiv \rho_\beta  \omega_{\alpha_1  \cdots \alpha_p},
  \]
  and that $\left(\rho_\beta\right)_\beta$ is a partition of unity.
\end{proof}


\begin{lemma}
\label{lemma:SES}
    The sheaf cochain complex 
    \[
        \begin{diagram}
            \node{0} \arrow{e,t}{} \node{\R} \arrow{e,t}{} \node{\cA^0} \arrow{e,t}{d} \node{\cA^1} \arrow{e,t}{d}\node{\cA^2} \arrow{e,t}{d} \node{\cdots}
        \end{diagram}
    \]
    is exact, that is it splits into short exact sequences
    \begin{gather*}
        \begin{diagram}
            \node{0} \arrow{e,t}{} \node{\R} \arrow{e,t}{} \node{\cA^0} \arrow{e,t}{d} \node{\cZ^1} \arrow{e,t}{}\node{0}
        \end{diagram}\\
        \vdots\\
        \begin{diagram}
            \node{0} \arrow{e,t}{} \node{\cZ^{p}} \arrow{e,t,J}{} \node{\cA^p} \arrow{e,t}{d} \node{\cZ^{p+1}} \arrow{e,t}{}\node{0}
        \end{diagram}\\
        \vdots
    \end{gather*}
\end{lemma}
\begin{proof}
    By definition, the cochain complex is exact iff it is exact on stalks. Thus, it suffices to prove that for every $x \in M$ there exists $ x\in U \subset M$ open and sufficiently small, s.t.\ 
    \[
            \begin{diagram}
                \node{\cdots} \arrow{e,t}{} \node{\cA^{p-1}(U)} \arrow{e,t}{d} \node{\cA^p(U)} \arrow{e,t}{d}\node{\cA^{p+1}(U)} \arrow{e,t}{d} \node{\cdots}
            \end{diagram}   
    \]
    is exact as a cochain complex of Abelian groups. However, this is exactly the content of the classical Poincar\'e Lemma, see for example \cite[Theorem~17.14]{Lee12}. If $U$ is diffeomorphic to a star-shaped open subset of $\R^n$, e.g.\ diffeomorphic via a chart to an open ball, then every closed differential form is exact. The second statement follows then as $\ker\left(d^p\big|_U\right) = \cZ^p(U) = \im\left(d^{p+1}\big|_U\right)$.
\end{proof}


The following important result allows one to relate the $\check{\mathrm{C}}$ech cohomology group of $M$ with the de Rham cohomology group and to have a criterion for when a cover is fine enough.


\begin{theorem}[De Rham's Theorem]
    There exists an isomorphism
    \[ 
       \Cech(M ; \R) \cong \Rham(M;\R). 
    \]
    and in particular 
    \[
       \Cech(M ; \R) \cong   \Cech(\mathcal U ; \R)
    \]
    for any open cover wherein each set is contractible.
\end{theorem}
\begin{proof}
  From the short exact sequences in \autoref{lemma:SES} we can deduce the existence of the following corresponding long exact sequences on cohomology: for all $p,q \in \N$
  \begin{align*}
    \cdots &\longrightarrow H^{p-1}(M, \cA^{q-1}) \longrightarrow H^{p-1}(M, \cZ^{q})\longrightarrow \\
    & \longrightarrow H^{p}(M, \cZ^{q-1})\longrightarrow H^{p}(M, \cA^{q-1})\longrightarrow \cdots.
  \end{align*}
  For $p > 1$ the leftmost and rightmost terms are zero thus 
  \[
    H^{p-1}(M, \cZ^{q}) \cong  H^{p}(M, \cZ^{q-1}).
  \]
  This gives us the following sequence of isomorphisms
  \begin{align*}
    \Cech^p(M, \R) &\coloneqq H^p(M,\R) \cong H^{p-1}(M,\cZ^{1}) \cong\\
    & \cong  H^{p-2}(M,\cZ^{2}) \cong \cdots \cong  H^{1}(M,\cZ^{p-1}).
  \end{align*}
  For $p = 1$ (in the above sense) we get the following exact sequence
  \[
      H^0(M, \cA^{p-1}) \longrightarrow  H^0(M, \cZ^{p}) \longrightarrow H^1(M, \cZ^{p-1}) \longrightarrow 0.  
  \]
  Since the $0^{\text{th}}$ order cohomology is simply the space of global sections the above is isomorphic
  \[
      \cA^{p-1}(M) \stackrel{d}{\longrightarrow} \cZ^{p}(M) \longrightarrow H^1(M, \cZ^{p-1}) \longrightarrow 0.  
  \] 
  By exactness it follows therefore that 
  \[
      H^1(M, \cZ^{p-1}) \cong   \cZ^{p}(M) /  d \cA^{p-1}(M) \eqqcolon \Rham^p(M, \R).
  \]
  This proves the assertion.
\end{proof}





\bibliographystyle{Martin}
\bibliography{./refs}




\end{document}